 \section{Introduction and Background}
\label{intro}
Information Centric Networking(ICN) is an approach of redesigning the
Internet as a network architecture that focuses on \emph{content} (or
named data) rather than \emph{hosts}(as in today's Internet). Various
ongoing future Internet projects like XIA\cite{xia}, NDN\cite{ndn}
incorporate the idea of ICN. The reason for this shift of focus is that
most of the applications running over the Internet today are content
oriented. For example, when you access youtube.com in your browser
window, you are more interested in watching the video than connecting
to youtube's video server. The nodes in ICN talk in terms of
content. So, these architectures have notions of \emph{Content
  Request} and \emph{Content Response} (See table
\ref{table:content}).

\begin{table}
  \begin{center}
    \begin{tabular} {c c c}
      Architecture & Content Request & Content Response \\
      \hline
      NDN & Interest Packet & Data Packet \\
      XIA & CID Request packet & CID Response Packet \\
    \end{tabular}
    \caption{Notion of Content Request and Content Response in various
    FIA proposals}
    \label{table:content}
  \end{center}
\end{table}

\subsection{Content Caching in ICN}
\label{intro_caching}
One of the key features of these architectures is ``in-network
caching''. The \emph{Content Response} objects enable in-network
caching on routers (or even on dedicated cache servers). In-network
caching optimizes bandwidth consumption, reduces network congestion
and provides fast fetching for the popular content.


\subsection{Open Problems related to Content Caching in ICN}
\label{caching_problems}
In the context of in-network-caching, much has been researched about
content popularity determination, cache eviction policies, cache
topologies. But, there is not much literature on the effect of
prevalence of content caches on other components of the Internet
(transport protocols / naming service). In other words, we do not
exactly know how the existing network protocols should be modified to
effectively utilize the caching infrastructure. Section \ref{icn_dns}
talks about the impact on DNS and the questions that need to be
answered. Section \ref{priv} talks about trade-offs between privacy
and caching performance. Section \ref{apis} looks at what APIs
underlying caching architecture can provide to the applications
programmers to effectively use caches. Note that in order to limit the
proposal size, I have only included 3 problem areas. I intend to look
at many more.

As this study involves understanding the impact of content caching on
layers from top to bottom, we call it a ``Vertical Slice
Approach''. However, given the time constraints, it might not be
possible to study each and every layer. In the initial period, I will
spend considerable amount of time on identifying which problems to
study (See section \ref{timetable} for details).

%% \begin{itemize}
%% \item How would webpages look like
%% \item How would name service work
%% \item Tradeoffs between privacy and performance
%% \item Content routing
%% \end{itemize}
