\section{Goals}
As mentioned in the previous section, the impact of content caches on
network architecture design has not been well studied. Studying this
would be the ultimate goal of my thesis. In my thesis, I not only wish
to find answers to some of the challenging questions but also come up
with interesting research challenges that can be posed to the ICN
community. In order to establish some context to the questions
mentioned in \ref{icn_dns}, \ref{priv} and \ref{apis}, I will try to
study these problems with respect to \emph{eXpressive Internet
  Architecture}\cite{xia} which is an ongoing research project at CMU.
I don't want to restrict myself to the questions mentioned in this
proposal. Rather, I am excited to come up with new research
questions. Following sections list a few candidate questions that can
be looked at.

\subsection{DNS and ICN}
\label{icn_dns}
Traditionally, DNS has been responsible for providing IP addresses for
names. Note that these names have been identifying hosts (typically
servers running that service). However, in ICN, DNS has much larger
role to play: it must provide network addresses for content names as
well as host names. Since content names outnumber host names by
several orders of magnitude, we most probably will need to redesign
existing DNS design. For example, to save DNS overhead, wherever
possible, content publisher might provide Network IDs for some content
objects (e.g., in case of web pages where publisher needs to embed
links to other web pages). However, this means we are essentially
moving name look up from client to the publisher. This is quite
opposite to today's DNS mechanism and hence needs in-depth study.

\subsection{Privacy and Performance}
\label{priv}
Privacy comes at a cost and that cost has been well captured in
\cite{cost_https}. In context of ICN caching, the cost of privacy
might be as severe as not being able to cache content at all. For
example, caching becomes impossible when client opens a secure tunnel
with the end server (as in TLS). Also, some of the recent proposals in
the networking community demand the use of HTTPS by default. Such
proposals can prove to be detrimental to caching in ICN. Since
security can not be compromised, ICN caching needs to accommodate the
existence of TLS like security protocols. One possible way achieve
that by letting clients establish tunnels with in-network caches
instead of final server. But, what might its implications be? Aren't
we adding extra processing overhead? In the context of privacy, we
need to find answers to questions like these.

\subsection{Caching APIs}
\label{apis}
Today, we have the notion of sockets as end points of communication.
But, is that sufficient in ICN since we care about data more? For
example, in ICN, since caches exist everywhere, we could give
abilities to the applications to pre-allocate a cache \emph{slice} on
router. But, there are many pros and cons in doing this. So, the
problem of defining Caching APIs aims at identifying the trade-offs
and come up with a set of APIs that we could provide to applications.
