\abstract

The World Wide Web (WWW) has arguably been the most popular
application of the Internet for years. Over a period of time, it has
developed over the principles of host-centric IP internet. However,
the limitations of today’s host-centric IP internet have motivated
many future internet architectures that are centered around alternate
principals such as content and services.

In this thesis, we study the WWW and propose features needed by such
clean slate future internet architectures that can benefit the
WWW. The features that we propose are then implemented on eXpressive
Internet Architecture (XIA) - a candidate future internet
architecture.

Most of the clean slate architectures proposed so far revolve around
``a'' principal giving rise to networking infrastructural styles such
as content-centric-networking or service-oriented networking. XIA
argues that elevating one principal above others limits the ability to
communicate with the other principals. Thus, XIA inherently supports
co-existence of multiple communication principals.

The WWW relies on a reliable transport layer for content delivery. We
see how these coexisting principals cooperate to provide a new
reliable content delivery architecture that offers content caching and
reliable content transport as services. Despite being offered as
services, we still maintain primary features offered by a content
centric network such as in-network caching and content routing.

We define a new principal type that allows fetching content by human
readable names rather than cryptographically secure
identifiers. Although human readable names are more convenient to for
the WWW, they are more vulnerable to security threats than
cryptographically secure identifiers. We address authenticity,
integrity issues raised by human readable names. We then define a
security model that allows endpoints as well as in-network devices to
perform integrity, authenticity checks in constant time.

To complete the story, we avoid the need of a huge naming system by
defining URL scheme that directly address es the content. Based on
these URL formats, the human readable identifiers and the new content
delivery system proposed, we model the World Wide Web over XIA.
