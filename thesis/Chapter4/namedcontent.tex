\chapter{Fetching Named Content in XIA}
\label{chap:namedcontent}

\section{Motivation}

In order to effectively utilize in-network caches, it is important
that the resources in the web are cacheable. Cacheability of the
resources really depends on its reusability. Static resources (objects
images, video chunks) are probably the best candidates for utilizing
in-network storage. But, the majority of web content falls in the
other two categories.

It can be argued that dynamic resources can benefit the least from of
in-network caches. The primary reason being that dynamic resources are
generated upon the request arrival and hence are highly
non-reusable. This, clients can thus

But the resources that are of interest to us are the multiform
resources that we saw in chapter \ref{chap:www}. Recall that these
resources exist in multiple forms at a same time or at different
times. Clients \emph{choose} a representation that suites their need
the best. They are reusable for the very reason that upon sending the
same request multiple times, they respond with the same content. What
can we do to make these multiform resources cacheable at the same time
preserve their multiformity?

\subsection{Human Readable Names for Multiform Resources}

We have seen that web relies on human readable names for identifying
multiform resources and in general all the web resources. Even though
cryptographic identifiers have inherent security properties, what
makes human readable names a preferred choice?

The first obvious reason is that the names are more
understandable. Users establish trust in human readable Identifiers
such as``facebook.com'' much easily than in random cryptic hex string
such as ``0ABF1866BD7182...''.

The second reason is that multiform resources change their
representation. For example, content associated with the resource
identified by ``http://www.cnn.com'' changes often. So, as time passes
multiform web resources change their representation. Also, time is not
the only dimension along which these resources change their
representation. Another such dimension could be UserAgent. With the
advent of smart-devices, number of platforms from which web resources
are accessed has been increasing like never before. That poses us with
the challenge of presenting the same resource in different forms based
on the platform that the user is requesting the resource from.

CIDs have the disadvantage that an identifier can only point to a
particular representation. It also implies that it is not possible for
us to allocate an identifier for content that we don’t know in
advance. So, crypto IDs do not support the property of late binding.

\section{Locating Content using Human Readable Names}

In the last section we saw the reasons why the world wide web
identifies and locates resources by human readable names. Let’s see
the possible options to locate content using human readable names in
eXpressive Internet Architecture.

\begin{itemize}
\item Keep a DNS like mapping system that maps names to CIDs
\item Locate content directly by human readable names
\end{itemize}

\subsection{Problems with DNS like mapping system}

In this approach, we would need a naming system that maps human
readable names to CIDs. Users first query the naming system to get the
CID and then request the CID. The naming systems could well be
established at organizational levels like today. Although scalable,
this approach suffers from following issues.

Size of the mapping system: Firstly, the size of the mappings that the
system needs to maintain is directly proportional to number of
cacheable objects that the publisher has as opposed to number of hosts
in today’s Internet. It is evident that number of objects outnumbers
number of hosts by orders of magnitude. So, we would need really huge
mapping system.

Number of roundtrips: In order to fetch a CID for a particular name,
the web user would need to make several roundtrips to and from the
naming system. Depending upon the model used by the naming system, the
consumer could take from one to several roundtrips before it can know
the CID corresponding to the content name.

Security Issues: Once the consumer receives CID corresponding to a
human readable name why would the consumer believe that mapping
between the human readable name and the CID is authentic? Hence, the
content retrieval architecture must address these security issues
somewhere. However, that loses the whole point of having cryptographic
identifiers. Remember that one important property of CIDs is that they
are self certifying.

Because of the issues mentioned above, it becomes unwise to use a
naming system for locating content by name. Therefore, we take the
second approach of directly addressing content by human readable
identifiers. The following sections describe the new principal type
that we define and address the security concerns raised by human
readable names.

\section{nCID Principal Type}

\subsection{Definition}

In order to effectively eliminate a huge naming system and support the
multiform content on the world wide web, we define a new principal
type which allows locating content directly by human readable
names. Just like CIDs, the content requested by the consumer using
nCID is retrieved from anywhere in the network. The XID type nCID is
defined as follows:

\begin{verbatim}
nCID = hash(Human Readable Name + Publisher's Public Key Fingerprint)
\end{verbatim}
nCID is thus content with a human readable name that is verified by a
publisher.

\subsection{Semantics}

\begin{table}
  \begin{center}
    \begin{tabular}
      {p{2in}  p{3in}}
      API & Details \\
      \hline
      XputNamedContent(Buffer, Name, Certificate, Signature) &
      Publishes a named content chunk with the corresponding signature
      and the certificate\\
      XgetNamedChunk(Name, Certificate) & Fetches a name content chunk
      and verifies the authenticity and integrity using the public key
      certificate\\
    \end{tabular}
  \end{center}
  \caption{nCID Principal Semantics}
  \label{tab:ncidapis}

\end{table}

The nCID principal allows users to retrieve content identified by
human readable names from anywhere in the network. \ref{tab:ncidapis}
shows the APIs that nCID supports. Similar to CIDs, sending content
request for nCID type using \texttt{getNamedContent()} initiates a
transport session with any in-network cache or the original
publisher. This reliable transport session is then used to deliver
content to end consumers.

The \texttt{publishNamedContent(name, signature, public key
  fingerprint)} call tells the network that the content identified by
a human readable name is available with the publisher. It is verified
by a private key corresponding to the public key passed as an
argument. Xcache expects that the signature passed an argument is
generated in a certain way (we elaborate upon it in the following
section). The public key fingerprint and the signature is used by
in-network devices and the endpoints to perform security checks.

\section{Security Issues with Named Content}

\subsection{nCID Security Requirements}

XIA’s intrinsic security requirement makes it mandatory for the XIA
prinicpals to provide integrity and authenticity for the communication
operation. CIDs provide the integrity and authentication by defining
identifiers as the hash of the content so that when the consumer
receives the content and the CID, it has a reason to believe that the
content has not be tampered with and has been received from an
authentic publisher.

We argue that satisfying following four requirements provides exactly
these guarantees for nCID principal type.

\begin{enumerate}
\item{Ability to Verify Binding between Name and Content} \\
The consumer and in-network devices should have a reason to believe
that the content and name are tied together. CIDs provide this
guarantee inherently. For nCIDs, we rely on public key infrastructure
to provide the guarantees.
\item{Ability to Verify Content Integrity} \\
The consumer should have a reason to believe that the content has not
been tampered with.
\item{Ability for Caches to Verify All These at Minimum Cost}\\
The in-network caches need to perform these security checks before
they can cache a copy of the object. Applications can use different
trust models. It would be inefficient for in-network caches to verify
authenticity, integrity properties of the content chunk based on the
application specific trust model. So, we need an application agnostic
security model to work with.
\item{Protection against Content Poisoning} \\
Since there is no absolute binding between the name and the content,
an attacker can claim that certain content can belong to a particular
name. Our security model must prevent such an action.
\end{enumerate}
\subsection{Security Model}

Previous work argues that in order to provide security guarantees it
is sufficient to bind any two pairs between name, content and
publisher. We choose to bind name-content and name-publisher pairs.

In order to understand how our security model functions let us look at
some important chunk headers that nCID chunk contains.

\begin{figure}
  \begin{center}
    \includegraphics[scale = 0.75]{ncid_chunk_structure}
    \caption{nCID Content Chunk Structure}
    \label{fig:ncid_chunk_structure}
  \end{center}
\end{figure}

Name-content binding is provided by the digital signature that the
publisher generates at publish time. Name-publisher binding is
provided by the nCID itself. Following equations show how these
bindings are created by the publisher.
\begin{center}
\texttt{\textbf{nCID = hash(Content Name, Publisher's Public Key Fingerprint)}}\\
\texttt{\textbf{Signature = Encrpyt\_with\_Publisher's\_Private\_Key (Content Name, Content Data)}}\\
\end{center}
As long as the consumer knows about the name of the content and the
original publisher's public key fingerprint, it can always generate a
content request. So, we expect that some high level entity (such as a
TLS connection) delivers these two parameters to the end consumer. In
the next chapter about URLs, we see how sophisticated URLs for nCID
can be used to serve this information.

nCID intrinsic security checking is a two-step process in contrast to
one-step process in case of CIDs. The entity that needs to verify
authenticity and integrity of the content chunk must first fetch the
public key that was used to verify the content. The content chunk
contains a pointer to the public key chunk.

Once the public key has been received, the consumer checks if nCID
matches the hash of name and publisher’s public key fingerprint. It
then decrypts the signature with the same public key. The decrypted
signature is matched against name-content pair. If both these checks
succeed, then consumer can safely believe that content is authentic
and has not been tampered with.

No matter what trust model the application uses, all the in-network
devices need to perform only one public key fetch and the two checks
for nCID and signature. Thus, the time required for verifying security
properties is constant. Besides, it requires at most only one content
chunk fetch. This satisfies our third requirement of verifying
security properties at minimum cost.

\subsection{Protection against Content Poisoning}
Content poisoning is an attack in which the attacker claims that a
certain malicious content belongs to certain name that has already
been published in the network. \ref{fig:content_poisoning} shows the
motivating example.

\begin{figure}
  \begin{center}
    \includegraphics[scale=0.5]{content_poisoning}
  \caption{Content Poisoning}
  \label{fig:content_poisoning}
  \end{center}
\end{figure}

In this example, the original publisher facebook.com has published a
nCID chunk named ``fb.com/cmu''. Facebook has verified the chunk and
put its signature in the chunk header. Now, an attacker wants to claim
that certain spoofed content actually belongs to the name
fb.com/cmu. What possible options he has to fill in signature and the
public key?

\begin{itemize}

\item{\textbf{Option 1: Public Key = Facebook’s Public Key}} \\
If attacker uses facebook’s public key itself then he cannot generate
the corresponding signature. So, this option is not really possible.

\item{\textbf{Option 2: Use my own key}}\\
Let’s say the attacker uses his own key to generate the signature. In
that case, even though attacker successfully plants a spoofed
signature in the chunk he breaks the nCID check.

\end{itemize}

So we have seen that the attacker has no way of successfully filling
in the signature and public key field pairs. Content poisoning is thus
not possible in our security model.

\section{Conclusion}
We defined a new content principal for XIA that allows us to directly
address the content by human readable names. We argued that such a
principal best suites the ``multiform web resources''. However, such a
content addressing system faces the issue of content authenticity and
content integrity. We defined security models that address these
authenticity and integrity issues to provide us an alternate secure
content principal.
