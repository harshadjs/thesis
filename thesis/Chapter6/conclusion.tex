\chapter{Conclusion}
\label{sec:conclusion}

The old implementation of the content principal did not support
reliable transport of content objects. The ability to reliably deliver
content objects is crucial for the World Wide Web. We thus implemented
content principal implementation to an application called
XcacheD. Moving content principal implementation to an application was
not a trivial task. It involved redefining interfaces with the network
stack. We carefully studied possible approaches and implemented the
content principal handling in Xcached application.

In order to fetch content objects directly by human readable names, we
defined a new XIA principal type - nCID. We addressed the authenticity
and integrity issues of the new principal type. The new principal type
allowed the clients to avoid name lookups for fetching content chunks.

We classified the web resources into three categories: static
resources, dynamic resources and multiform resources. XIA's different
communication principals supported these different web resources. We
argued that the static resources can be well represented with CIDs,
the dynamic resources can be well represented with SIDs and the
multiform resources can be well represented with nCIDs.

In order to effectively reference these various resources, we defined
a URL scheme. We defined a generic URL scheme to map any XIA address
to serialized character string. This scheme allowed us to represent
addresses of static and dynamic resources. We then defined URL format
for nCID principal that allowed us to address the multiform resources.

With the above contributions, we modeled the World Wide Web on
eXpressive Internet Architecture.
