\chapter{World Wide Web}
\label{chap:www}
The World Wide Web is arguably the most important application of the
Internet. The World Wide Web is an information space that allows
exchanging information objects called web resources between hosts. It
was invented by English scientist Tim Berners-Lee in 1989.

The web is so popular that the term is often used interchangeably with
the Internet itself. However, these two are not the same. The Internet
is a giant network of interconnected computers identified by IP
address. Whereas, the web is a collection of web resources such as
documents, videos, images that these interconnected computers can
network. Essentially, the web runs \emph{on top of} the Internet.

In this chapter we will take a closer look at the important components
of the web.

\section{Universal Resource Identifiers}
The web resources are uniquely identified by something called as
Universal Resource Identifiers or URIs. URI is a string of characters
that can identify a web resource uniquely. This unique identification
provides the network entities a way to identify and therefore
\emph{request} as well as \emph{serve} a resource. RFC 2396 formally
defined the format of URI. The definition was later refined by RFC
3986. The simplified definition of the URI is as follows.
\begin{center}
  \begin{verbatim}
  URI         = scheme : hierarchical-part [ ? query ] [ # fragment ]
  hierarchical-part   = // authority path
                         / path
\end{verbatim}
\end{center}
\begin{itemize}
  \item{\emph{Scheme:}} Examples of popular schemes are HTTP, FTP,
    mailto etc.
  \item{\emph{Hierarchical Part:}} Location of a web resource within
    some logical hierarchy. Often, this part is formed by combining
    the host (a registered name or an IPv4 address) and hierarchical
    path (similar to UNIX file system paths).
  \item{\emph{Query:}} Traditionally consists of key-value pairs.
  \item{\emph{Fragment:}} A character string that identifies a
    fragment in the resource. For example, a section in an article.
\end{itemize}
The example of a URI given in RFC 3986 is as follows.
\begin{verbatim}
foo://example.com:8042/over/there?name=ferret#nose
\_/   \______________/\_________/ \_________/ \__/
 |           |            |            |        |
scheme     authority       path        query   fragment
\end{verbatim}
URL (Universal Resource Locator) is the most commonly used form of URI
in the World Wide Web. In addition to uniquely identifying a web
resource, URL also provides a way to locate the resource. Essentially,
URL identifies a web resource by its network location.
\section{Hypertext Transfer Protocol}
The primary method used for publishing and retrieving these web
resources is Hypertext Transfer Protocol (HTTP). Although HTTP is one
of many Internet communication protocols, the web resources are
usually accessed via HTTP. HTTP is a request-response type of
protocol. The HTTP clients or web clients request resources by sending
a \texttt{HTTP GET} request to entities serving these resources. The
entities that serve the web resources are called HTTP servers or Web
Servers.

\subsection{HTTP Session}
HTTP Session is a sequence of request-response transactions. The HTTP
client initiates a reliable transport session (a TCP session) with a
HTTP server listening on a particular predefined port. The port number
used by HTTP servers is usually 80. Once the session has been
established, the client then sends a HTTP request to the
server. Server responds back with a status line such as \texttt{HTTP
  200 OK} and the message which contains the actual object.

\subsection{HTTP Metadata}
HTTP requests and responses are coupled with metadata that describe
these requests and responses. The HTTP metadata describes one of the
following:
\begin{itemize}
\item{HTTP Session - Describes the current HTTP session. For example,
  metadata ``connection'' describes if the web server should
  terminate the current HTTP session after this request / response or not.}
\item{The Web Server - Describes the web server itself. For example,
  the metadata ``Server'' gives the name of the server.}
\item{The Web Client - Describes the web client. For example. the
  metadata ``UserAgent'' is the user software that is requesting the resource}
\item{Web Resource - Describes the web resource being served. For
  example, ``Content-Length'' is the length of the resource being served.}
\item{HTTP Services: Caching, Content Negotiation - HTTP supports
  various services such caching at a web proxy or negotiation the form
  of the content. Some of the metadata fields describe the attributes
  of these services.}
\end{itemize}

\subsection{HTTP Methods}
HTTP supports different request types. The following are the different
types of methods that HTTP supports: \texttt{GET, HEAD, POST, PUT,
  DELETE, TRACE, OPTIONS, CONNECT.} The HTTP method that is
responsible for fetching a web resource is \texttt{GET}. The HTTP
method \texttt{HEAD} is used when the web client is only interested
in fetching the metadata associated with the web resource and does not
worry about the actual web resource.

\section{Web Resources}
The resources in the web can be categorized in the following three
types.

\begin{itemize}
  \item{Static Resources: Static resources are the web resources do
    not change their form over a long time or based on the metadata
    presented in the request. An example of a static resource is a
    static image. A peculiar characteristic of a static resource is
    that the URL often points to a file name. For example, the URL
    \texttt{http://upload.wikimedia.org/wikimedia/google.jpg} points
    to a \texttt{JPEG} file.}
  \item{Dynamic Resources: A Web resource that is generated upon the
    request from a web client is called a Dynamic
    Resource. Personalized Facebook homepage is an example of a
    dynamic resource.}
  \item{Multiform Resources: Multiform resources are midway between
    static and dynamic resources. A multiform web resource exists in
    multiple different forms. Based on the context presented by the
    requester, it changes its form. A webpage that changes its form
    based on the UserAgent is an example of multiform resource. The
    CNN homepage \texttt{http://www.cnn.com} is another such
    example. The CNN homepage changes its form as and when news
    arrive. But the URL that is used for retrieving the resource is
    still the same.}
\end{itemize}

\section{Conclusion}
In this chapter, we studied the background the World Wide Web that is
relevant to the contribution of the thesis. The following chapter
talks about future of Internet research and a candidate future
Internet architecture - eXpressive Internet Architectire (XIA) in
which we implement our ideas.
